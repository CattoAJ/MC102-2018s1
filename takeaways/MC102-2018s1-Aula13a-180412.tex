
% Default to the notebook output style

    


% Inherit from the specified cell style.




    
\documentclass[11pt]{article}

    
    
    \usepackage[T1]{fontenc}
    % Nicer default font (+ math font) than Computer Modern for most use cases
    \usepackage{mathpazo}

    % Basic figure setup, for now with no caption control since it's done
    % automatically by Pandoc (which extracts ![](path) syntax from Markdown).
    \usepackage{graphicx}
    % We will generate all images so they have a width \maxwidth. This means
    % that they will get their normal width if they fit onto the page, but
    % are scaled down if they would overflow the margins.
    \makeatletter
    \def\maxwidth{\ifdim\Gin@nat@width>\linewidth\linewidth
    \else\Gin@nat@width\fi}
    \makeatother
    \let\Oldincludegraphics\includegraphics
    % Set max figure width to be 80% of text width, for now hardcoded.
    \renewcommand{\includegraphics}[1]{\Oldincludegraphics[width=.8\maxwidth]{#1}}
    % Ensure that by default, figures have no caption (until we provide a
    % proper Figure object with a Caption API and a way to capture that
    % in the conversion process - todo).
    \usepackage{caption}
    \DeclareCaptionLabelFormat{nolabel}{}
    \captionsetup{labelformat=nolabel}

    \usepackage{adjustbox} % Used to constrain images to a maximum size 
    \usepackage{xcolor} % Allow colors to be defined
    \usepackage{enumerate} % Needed for markdown enumerations to work
    \usepackage{geometry} % Used to adjust the document margins
    \usepackage{amsmath} % Equations
    \usepackage{amssymb} % Equations
    \usepackage{textcomp} % defines textquotesingle
    % Hack from http://tex.stackexchange.com/a/47451/13684:
    \AtBeginDocument{%
        \def\PYZsq{\textquotesingle}% Upright quotes in Pygmentized code
    }
    \usepackage{upquote} % Upright quotes for verbatim code
    \usepackage{eurosym} % defines \euro
    \usepackage[mathletters]{ucs} % Extended unicode (utf-8) support
    \usepackage[utf8x]{inputenc} % Allow utf-8 characters in the tex document
    \usepackage{fancyvrb} % verbatim replacement that allows latex
    \usepackage{grffile} % extends the file name processing of package graphics 
                         % to support a larger range 
    % The hyperref package gives us a pdf with properly built
    % internal navigation ('pdf bookmarks' for the table of contents,
    % internal cross-reference links, web links for URLs, etc.)
    \usepackage{hyperref}
    \usepackage{longtable} % longtable support required by pandoc >1.10
    \usepackage{booktabs}  % table support for pandoc > 1.12.2
    \usepackage[inline]{enumitem} % IRkernel/repr support (it uses the enumerate* environment)
    \usepackage[normalem]{ulem} % ulem is needed to support strikethroughs (\sout)
                                % normalem makes italics be italics, not underlines
    

    
    
    % Colors for the hyperref package
    \definecolor{urlcolor}{rgb}{0,.145,.698}
    \definecolor{linkcolor}{rgb}{.71,0.21,0.01}
    \definecolor{citecolor}{rgb}{.12,.54,.11}

    % ANSI colors
    \definecolor{ansi-black}{HTML}{3E424D}
    \definecolor{ansi-black-intense}{HTML}{282C36}
    \definecolor{ansi-red}{HTML}{E75C58}
    \definecolor{ansi-red-intense}{HTML}{B22B31}
    \definecolor{ansi-green}{HTML}{00A250}
    \definecolor{ansi-green-intense}{HTML}{007427}
    \definecolor{ansi-yellow}{HTML}{DDB62B}
    \definecolor{ansi-yellow-intense}{HTML}{B27D12}
    \definecolor{ansi-blue}{HTML}{208FFB}
    \definecolor{ansi-blue-intense}{HTML}{0065CA}
    \definecolor{ansi-magenta}{HTML}{D160C4}
    \definecolor{ansi-magenta-intense}{HTML}{A03196}
    \definecolor{ansi-cyan}{HTML}{60C6C8}
    \definecolor{ansi-cyan-intense}{HTML}{258F8F}
    \definecolor{ansi-white}{HTML}{C5C1B4}
    \definecolor{ansi-white-intense}{HTML}{A1A6B2}

    % commands and environments needed by pandoc snippets
    % extracted from the output of `pandoc -s`
    \providecommand{\tightlist}{%
      \setlength{\itemsep}{0pt}\setlength{\parskip}{0pt}}
    \DefineVerbatimEnvironment{Highlighting}{Verbatim}{commandchars=\\\{\}}
    % Add ',fontsize=\small' for more characters per line
    \newenvironment{Shaded}{}{}
    \newcommand{\KeywordTok}[1]{\textcolor[rgb]{0.00,0.44,0.13}{\textbf{{#1}}}}
    \newcommand{\DataTypeTok}[1]{\textcolor[rgb]{0.56,0.13,0.00}{{#1}}}
    \newcommand{\DecValTok}[1]{\textcolor[rgb]{0.25,0.63,0.44}{{#1}}}
    \newcommand{\BaseNTok}[1]{\textcolor[rgb]{0.25,0.63,0.44}{{#1}}}
    \newcommand{\FloatTok}[1]{\textcolor[rgb]{0.25,0.63,0.44}{{#1}}}
    \newcommand{\CharTok}[1]{\textcolor[rgb]{0.25,0.44,0.63}{{#1}}}
    \newcommand{\StringTok}[1]{\textcolor[rgb]{0.25,0.44,0.63}{{#1}}}
    \newcommand{\CommentTok}[1]{\textcolor[rgb]{0.38,0.63,0.69}{\textit{{#1}}}}
    \newcommand{\OtherTok}[1]{\textcolor[rgb]{0.00,0.44,0.13}{{#1}}}
    \newcommand{\AlertTok}[1]{\textcolor[rgb]{1.00,0.00,0.00}{\textbf{{#1}}}}
    \newcommand{\FunctionTok}[1]{\textcolor[rgb]{0.02,0.16,0.49}{{#1}}}
    \newcommand{\RegionMarkerTok}[1]{{#1}}
    \newcommand{\ErrorTok}[1]{\textcolor[rgb]{1.00,0.00,0.00}{\textbf{{#1}}}}
    \newcommand{\NormalTok}[1]{{#1}}
    
    % Additional commands for more recent versions of Pandoc
    \newcommand{\ConstantTok}[1]{\textcolor[rgb]{0.53,0.00,0.00}{{#1}}}
    \newcommand{\SpecialCharTok}[1]{\textcolor[rgb]{0.25,0.44,0.63}{{#1}}}
    \newcommand{\VerbatimStringTok}[1]{\textcolor[rgb]{0.25,0.44,0.63}{{#1}}}
    \newcommand{\SpecialStringTok}[1]{\textcolor[rgb]{0.73,0.40,0.53}{{#1}}}
    \newcommand{\ImportTok}[1]{{#1}}
    \newcommand{\DocumentationTok}[1]{\textcolor[rgb]{0.73,0.13,0.13}{\textit{{#1}}}}
    \newcommand{\AnnotationTok}[1]{\textcolor[rgb]{0.38,0.63,0.69}{\textbf{\textit{{#1}}}}}
    \newcommand{\CommentVarTok}[1]{\textcolor[rgb]{0.38,0.63,0.69}{\textbf{\textit{{#1}}}}}
    \newcommand{\VariableTok}[1]{\textcolor[rgb]{0.10,0.09,0.49}{{#1}}}
    \newcommand{\ControlFlowTok}[1]{\textcolor[rgb]{0.00,0.44,0.13}{\textbf{{#1}}}}
    \newcommand{\OperatorTok}[1]{\textcolor[rgb]{0.40,0.40,0.40}{{#1}}}
    \newcommand{\BuiltInTok}[1]{{#1}}
    \newcommand{\ExtensionTok}[1]{{#1}}
    \newcommand{\PreprocessorTok}[1]{\textcolor[rgb]{0.74,0.48,0.00}{{#1}}}
    \newcommand{\AttributeTok}[1]{\textcolor[rgb]{0.49,0.56,0.16}{{#1}}}
    \newcommand{\InformationTok}[1]{\textcolor[rgb]{0.38,0.63,0.69}{\textbf{\textit{{#1}}}}}
    \newcommand{\WarningTok}[1]{\textcolor[rgb]{0.38,0.63,0.69}{\textbf{\textit{{#1}}}}}
    
    
    % Define a nice break command that doesn't care if a line doesn't already
    % exist.
    \def\br{\hspace*{\fill} \\* }
    % Math Jax compatability definitions
    \def\gt{>}
    \def\lt{<}
    % Document parameters
    \title{Tuplas \\
           \small{MC102-2018s1-Aula13a-180412}
           }
    \author{Arthur J. Catto, PhD \\
            \small{ajcatto@g.unicamp.br}
            }
    \date{12 de abril de 2018}
    
    
    

    % Pygments definitions
    
\makeatletter
\def\PY@reset{\let\PY@it=\relax \let\PY@bf=\relax%
    \let\PY@ul=\relax \let\PY@tc=\relax%
    \let\PY@bc=\relax \let\PY@ff=\relax}
\def\PY@tok#1{\csname PY@tok@#1\endcsname}
\def\PY@toks#1+{\ifx\relax#1\empty\else%
    \PY@tok{#1}\expandafter\PY@toks\fi}
\def\PY@do#1{\PY@bc{\PY@tc{\PY@ul{%
    \PY@it{\PY@bf{\PY@ff{#1}}}}}}}
\def\PY#1#2{\PY@reset\PY@toks#1+\relax+\PY@do{#2}}

\expandafter\def\csname PY@tok@w\endcsname{\def\PY@tc##1{\textcolor[rgb]{0.73,0.73,0.73}{##1}}}
\expandafter\def\csname PY@tok@c\endcsname{\let\PY@it=\textit\def\PY@tc##1{\textcolor[rgb]{0.25,0.50,0.50}{##1}}}
\expandafter\def\csname PY@tok@cp\endcsname{\def\PY@tc##1{\textcolor[rgb]{0.74,0.48,0.00}{##1}}}
\expandafter\def\csname PY@tok@k\endcsname{\let\PY@bf=\textbf\def\PY@tc##1{\textcolor[rgb]{0.00,0.50,0.00}{##1}}}
\expandafter\def\csname PY@tok@kp\endcsname{\def\PY@tc##1{\textcolor[rgb]{0.00,0.50,0.00}{##1}}}
\expandafter\def\csname PY@tok@kt\endcsname{\def\PY@tc##1{\textcolor[rgb]{0.69,0.00,0.25}{##1}}}
\expandafter\def\csname PY@tok@o\endcsname{\def\PY@tc##1{\textcolor[rgb]{0.40,0.40,0.40}{##1}}}
\expandafter\def\csname PY@tok@ow\endcsname{\let\PY@bf=\textbf\def\PY@tc##1{\textcolor[rgb]{0.67,0.13,1.00}{##1}}}
\expandafter\def\csname PY@tok@nb\endcsname{\def\PY@tc##1{\textcolor[rgb]{0.00,0.50,0.00}{##1}}}
\expandafter\def\csname PY@tok@nf\endcsname{\def\PY@tc##1{\textcolor[rgb]{0.00,0.00,1.00}{##1}}}
\expandafter\def\csname PY@tok@nc\endcsname{\let\PY@bf=\textbf\def\PY@tc##1{\textcolor[rgb]{0.00,0.00,1.00}{##1}}}
\expandafter\def\csname PY@tok@nn\endcsname{\let\PY@bf=\textbf\def\PY@tc##1{\textcolor[rgb]{0.00,0.00,1.00}{##1}}}
\expandafter\def\csname PY@tok@ne\endcsname{\let\PY@bf=\textbf\def\PY@tc##1{\textcolor[rgb]{0.82,0.25,0.23}{##1}}}
\expandafter\def\csname PY@tok@nv\endcsname{\def\PY@tc##1{\textcolor[rgb]{0.10,0.09,0.49}{##1}}}
\expandafter\def\csname PY@tok@no\endcsname{\def\PY@tc##1{\textcolor[rgb]{0.53,0.00,0.00}{##1}}}
\expandafter\def\csname PY@tok@nl\endcsname{\def\PY@tc##1{\textcolor[rgb]{0.63,0.63,0.00}{##1}}}
\expandafter\def\csname PY@tok@ni\endcsname{\let\PY@bf=\textbf\def\PY@tc##1{\textcolor[rgb]{0.60,0.60,0.60}{##1}}}
\expandafter\def\csname PY@tok@na\endcsname{\def\PY@tc##1{\textcolor[rgb]{0.49,0.56,0.16}{##1}}}
\expandafter\def\csname PY@tok@nt\endcsname{\let\PY@bf=\textbf\def\PY@tc##1{\textcolor[rgb]{0.00,0.50,0.00}{##1}}}
\expandafter\def\csname PY@tok@nd\endcsname{\def\PY@tc##1{\textcolor[rgb]{0.67,0.13,1.00}{##1}}}
\expandafter\def\csname PY@tok@s\endcsname{\def\PY@tc##1{\textcolor[rgb]{0.73,0.13,0.13}{##1}}}
\expandafter\def\csname PY@tok@sd\endcsname{\let\PY@it=\textit\def\PY@tc##1{\textcolor[rgb]{0.73,0.13,0.13}{##1}}}
\expandafter\def\csname PY@tok@si\endcsname{\let\PY@bf=\textbf\def\PY@tc##1{\textcolor[rgb]{0.73,0.40,0.53}{##1}}}
\expandafter\def\csname PY@tok@se\endcsname{\let\PY@bf=\textbf\def\PY@tc##1{\textcolor[rgb]{0.73,0.40,0.13}{##1}}}
\expandafter\def\csname PY@tok@sr\endcsname{\def\PY@tc##1{\textcolor[rgb]{0.73,0.40,0.53}{##1}}}
\expandafter\def\csname PY@tok@ss\endcsname{\def\PY@tc##1{\textcolor[rgb]{0.10,0.09,0.49}{##1}}}
\expandafter\def\csname PY@tok@sx\endcsname{\def\PY@tc##1{\textcolor[rgb]{0.00,0.50,0.00}{##1}}}
\expandafter\def\csname PY@tok@m\endcsname{\def\PY@tc##1{\textcolor[rgb]{0.40,0.40,0.40}{##1}}}
\expandafter\def\csname PY@tok@gh\endcsname{\let\PY@bf=\textbf\def\PY@tc##1{\textcolor[rgb]{0.00,0.00,0.50}{##1}}}
\expandafter\def\csname PY@tok@gu\endcsname{\let\PY@bf=\textbf\def\PY@tc##1{\textcolor[rgb]{0.50,0.00,0.50}{##1}}}
\expandafter\def\csname PY@tok@gd\endcsname{\def\PY@tc##1{\textcolor[rgb]{0.63,0.00,0.00}{##1}}}
\expandafter\def\csname PY@tok@gi\endcsname{\def\PY@tc##1{\textcolor[rgb]{0.00,0.63,0.00}{##1}}}
\expandafter\def\csname PY@tok@gr\endcsname{\def\PY@tc##1{\textcolor[rgb]{1.00,0.00,0.00}{##1}}}
\expandafter\def\csname PY@tok@ge\endcsname{\let\PY@it=\textit}
\expandafter\def\csname PY@tok@gs\endcsname{\let\PY@bf=\textbf}
\expandafter\def\csname PY@tok@gp\endcsname{\let\PY@bf=\textbf\def\PY@tc##1{\textcolor[rgb]{0.00,0.00,0.50}{##1}}}
\expandafter\def\csname PY@tok@go\endcsname{\def\PY@tc##1{\textcolor[rgb]{0.53,0.53,0.53}{##1}}}
\expandafter\def\csname PY@tok@gt\endcsname{\def\PY@tc##1{\textcolor[rgb]{0.00,0.27,0.87}{##1}}}
\expandafter\def\csname PY@tok@err\endcsname{\def\PY@bc##1{\setlength{\fboxsep}{0pt}\fcolorbox[rgb]{1.00,0.00,0.00}{1,1,1}{\strut ##1}}}
\expandafter\def\csname PY@tok@kc\endcsname{\let\PY@bf=\textbf\def\PY@tc##1{\textcolor[rgb]{0.00,0.50,0.00}{##1}}}
\expandafter\def\csname PY@tok@kd\endcsname{\let\PY@bf=\textbf\def\PY@tc##1{\textcolor[rgb]{0.00,0.50,0.00}{##1}}}
\expandafter\def\csname PY@tok@kn\endcsname{\let\PY@bf=\textbf\def\PY@tc##1{\textcolor[rgb]{0.00,0.50,0.00}{##1}}}
\expandafter\def\csname PY@tok@kr\endcsname{\let\PY@bf=\textbf\def\PY@tc##1{\textcolor[rgb]{0.00,0.50,0.00}{##1}}}
\expandafter\def\csname PY@tok@bp\endcsname{\def\PY@tc##1{\textcolor[rgb]{0.00,0.50,0.00}{##1}}}
\expandafter\def\csname PY@tok@fm\endcsname{\def\PY@tc##1{\textcolor[rgb]{0.00,0.00,1.00}{##1}}}
\expandafter\def\csname PY@tok@vc\endcsname{\def\PY@tc##1{\textcolor[rgb]{0.10,0.09,0.49}{##1}}}
\expandafter\def\csname PY@tok@vg\endcsname{\def\PY@tc##1{\textcolor[rgb]{0.10,0.09,0.49}{##1}}}
\expandafter\def\csname PY@tok@vi\endcsname{\def\PY@tc##1{\textcolor[rgb]{0.10,0.09,0.49}{##1}}}
\expandafter\def\csname PY@tok@vm\endcsname{\def\PY@tc##1{\textcolor[rgb]{0.10,0.09,0.49}{##1}}}
\expandafter\def\csname PY@tok@sa\endcsname{\def\PY@tc##1{\textcolor[rgb]{0.73,0.13,0.13}{##1}}}
\expandafter\def\csname PY@tok@sb\endcsname{\def\PY@tc##1{\textcolor[rgb]{0.73,0.13,0.13}{##1}}}
\expandafter\def\csname PY@tok@sc\endcsname{\def\PY@tc##1{\textcolor[rgb]{0.73,0.13,0.13}{##1}}}
\expandafter\def\csname PY@tok@dl\endcsname{\def\PY@tc##1{\textcolor[rgb]{0.73,0.13,0.13}{##1}}}
\expandafter\def\csname PY@tok@s2\endcsname{\def\PY@tc##1{\textcolor[rgb]{0.73,0.13,0.13}{##1}}}
\expandafter\def\csname PY@tok@sh\endcsname{\def\PY@tc##1{\textcolor[rgb]{0.73,0.13,0.13}{##1}}}
\expandafter\def\csname PY@tok@s1\endcsname{\def\PY@tc##1{\textcolor[rgb]{0.73,0.13,0.13}{##1}}}
\expandafter\def\csname PY@tok@mb\endcsname{\def\PY@tc##1{\textcolor[rgb]{0.40,0.40,0.40}{##1}}}
\expandafter\def\csname PY@tok@mf\endcsname{\def\PY@tc##1{\textcolor[rgb]{0.40,0.40,0.40}{##1}}}
\expandafter\def\csname PY@tok@mh\endcsname{\def\PY@tc##1{\textcolor[rgb]{0.40,0.40,0.40}{##1}}}
\expandafter\def\csname PY@tok@mi\endcsname{\def\PY@tc##1{\textcolor[rgb]{0.40,0.40,0.40}{##1}}}
\expandafter\def\csname PY@tok@il\endcsname{\def\PY@tc##1{\textcolor[rgb]{0.40,0.40,0.40}{##1}}}
\expandafter\def\csname PY@tok@mo\endcsname{\def\PY@tc##1{\textcolor[rgb]{0.40,0.40,0.40}{##1}}}
\expandafter\def\csname PY@tok@ch\endcsname{\let\PY@it=\textit\def\PY@tc##1{\textcolor[rgb]{0.25,0.50,0.50}{##1}}}
\expandafter\def\csname PY@tok@cm\endcsname{\let\PY@it=\textit\def\PY@tc##1{\textcolor[rgb]{0.25,0.50,0.50}{##1}}}
\expandafter\def\csname PY@tok@cpf\endcsname{\let\PY@it=\textit\def\PY@tc##1{\textcolor[rgb]{0.25,0.50,0.50}{##1}}}
\expandafter\def\csname PY@tok@c1\endcsname{\let\PY@it=\textit\def\PY@tc##1{\textcolor[rgb]{0.25,0.50,0.50}{##1}}}
\expandafter\def\csname PY@tok@cs\endcsname{\let\PY@it=\textit\def\PY@tc##1{\textcolor[rgb]{0.25,0.50,0.50}{##1}}}

\def\PYZbs{\char`\\}
\def\PYZus{\char`\_}
\def\PYZob{\char`\{}
\def\PYZcb{\char`\}}
\def\PYZca{\char`\^}
\def\PYZam{\char`\&}
\def\PYZlt{\char`\<}
\def\PYZgt{\char`\>}
\def\PYZsh{\char`\#}
\def\PYZpc{\char`\%}
\def\PYZdl{\char`\$}
\def\PYZhy{\char`\-}
\def\PYZsq{\char`\'}
\def\PYZdq{\char`\"}
\def\PYZti{\char`\~}
% for compatibility with earlier versions
\def\PYZat{@}
\def\PYZlb{[}
\def\PYZrb{]}
\makeatother


    % Exact colors from NB
    \definecolor{incolor}{rgb}{0.0, 0.0, 0.5}
    \definecolor{outcolor}{rgb}{0.545, 0.0, 0.0}



    
    % Prevent overflowing lines due to hard-to-break entities
    \sloppy 
    % Setup hyperref package
    \hypersetup{
      breaklinks=true,  % so long urls are correctly broken across lines
      colorlinks=true,
      urlcolor=urlcolor,
      linkcolor=linkcolor,
      citecolor=citecolor,
      }
    % Slightly bigger margins than the latex defaults
    
    \geometry{verbose,tmargin=1in,bmargin=1in,lmargin=1in,rmargin=1in}
    
    

    \begin{document}
    
    
    \maketitle
    
    

    
%    \begin{Verbatim}[commandchars=\\\{\}]
%{\color{incolor}In [{\color{incolor}1}]:} \PY{k+kn}{from} \PY{n+nn}{IPython}\PY{n+nn}{.}\PY{n+nn}{core}\PY{n+nn}{.}\PY{n+nn}{interactiveshell} \PY{k}{import} \PY{n}{InteractiveShell}
%        \PY{n}{InteractiveShell}\PY{o}{.}\PY{n}{ast\PYZus{}node\PYZus{}interactivity} \PY{o}{=} \PY{l+s+s2}{\PYZdq{}}\PY{l+s+s2}{all}\PY{l+s+s2}{\PYZdq{}}
%\end{Verbatim}


    \section{Tuplas}\label{tuplas}

    \subsection{O modelo}\label{o-modelo}

Uma \textbf{\emph{tupla}} é uma \textbf{\emph{sequência de objetos}},
\textbf{\emph{ordenada}}, \textbf{\emph{imutável}},
\textbf{\emph{iterável}} e \textbf{\emph{tipicamente heterogênea}}.

\emph{Tuplas} em Python são úteis para modelar o que outras linguagens
de alto nível geralmente chamam de \emph{records}. No entanto, ao invés
de ter um nome como um campo de um \emph{record}, cada elemento de uma
tupla é identificado por um \emph{índice} que indica sua posição na
estrutura. A indexação é a única forma de acesso aos elementos de uma
tupla.

Uma \emph{tupla} é sempre uma estrutura ad-hoc, isto é, definida no
momento de sua criação. Por isso, é difícil garantir que duas tuplas,
mesmo que relacionadas, tenham o mesmo número de campos e as mesmas
propriedades associadas a eles.

Assim, é fácil cometer erros difíceis de localizar, como a introdução de
campos extras ou a inversão de ordem entre campos. \emph{Namedtuples},
discutidas no final desta aula, representam uma extensão interessante
desse modelo e procuram atacar alguns desses problemas.

    Uma tupla é representada como uma sequência de valores (usualmente entre
parênteses) separados por vírgulas.

    \begin{Verbatim}[commandchars=\\\{\}]
{\color{incolor}In [{\color{incolor}2}]:} \PY{n}{auto} \PY{o}{=} \PY{p}{(}\PY{l+s+s1}{\PYZsq{}}\PY{l+s+s1}{Honda}\PY{l+s+s1}{\PYZsq{}}\PY{p}{,} \PY{l+s+s1}{\PYZsq{}}\PY{l+s+s1}{City}\PY{l+s+s1}{\PYZsq{}}\PY{p}{,} \PY{l+s+s1}{\PYZsq{}}\PY{l+s+s1}{DX 1.5}\PY{l+s+s1}{\PYZsq{}}\PY{p}{,} \PY{l+m+mi}{2016}\PY{p}{,} \PY{l+m+mf}{52500.0}\PY{p}{)}
        \PY{n}{auto}
\end{Verbatim}


\begin{Verbatim}[commandchars=\\\{\}]
{\color{outcolor}Out[{\color{outcolor}2}]:} ('Honda', 'City', 'DX 1.5', 2016, 52500.0)
\end{Verbatim}
            
    O número de elementos em uma \emph{tupla} é dado pela função
\texttt{len}.

    \begin{Verbatim}[commandchars=\\\{\}]
{\color{incolor}In [{\color{incolor}3}]:} \PY{n+nb}{len}\PY{p}{(}\PY{n}{auto}\PY{p}{)}
\end{Verbatim}


\begin{Verbatim}[commandchars=\\\{\}]
{\color{outcolor}Out[{\color{outcolor}3}]:} 5
\end{Verbatim}
            
    Podemos obter os elementos de uma \emph{tupla} por indexação direta,
iterando em um \texttt{for}, um \texttt{while} ou por fatiamento.

    \begin{Verbatim}[commandchars=\\\{\}]
{\color{incolor}In [{\color{incolor}5}]:} \PY{k}{for} \PY{n}{campo} \PY{o+ow}{in} \PY{n}{auto}\PY{p}{:}
            \PY{n+nb}{print}\PY{p}{(}\PY{n}{campo}\PY{p}{)}
\end{Verbatim}


    \begin{Verbatim}[commandchars=\\\{\}]
Honda
City
DX 1.5
2016
52500.0

    \end{Verbatim}

    \begin{Verbatim}[commandchars=\\\{\}]
{\color{incolor}In [{\color{incolor}6}]:} \PY{n}{i} \PY{o}{=} \PY{l+m+mi}{0}
        \PY{k}{while} \PY{n}{i} \PY{o}{\PYZlt{}} \PY{n+nb}{len}\PY{p}{(}\PY{n}{auto}\PY{p}{)}\PY{p}{:}
            \PY{n+nb}{print}\PY{p}{(}\PY{n}{auto}\PY{p}{[}\PY{n}{i}\PY{p}{]}\PY{p}{)}
            \PY{n}{i} \PY{o}{+}\PY{o}{=} \PY{l+m+mi}{1}
\end{Verbatim}


    \begin{Verbatim}[commandchars=\\\{\}]
Honda
City
DX 1.5
2016
52500.0

    \end{Verbatim}

    \begin{Verbatim}[commandchars=\\\{\}]
{\color{incolor}In [{\color{incolor}7}]:} \PY{n}{auto\PYZus{}2} \PY{o}{=} \PY{n}{auto}\PY{p}{[}\PY{p}{:}\PY{l+m+mi}{2}\PY{p}{]} \PY{o}{+} \PY{p}{(}\PY{l+s+s1}{\PYZsq{}}\PY{l+s+s1}{EX 1.5}\PY{l+s+s1}{\PYZsq{}}\PY{p}{,} \PY{l+m+mi}{2018}\PY{p}{,} \PY{l+m+mf}{69650.0}\PY{p}{)}
        \PY{n+nb}{print}\PY{p}{(}\PY{n}{auto\PYZus{}2}\PY{p}{)}
\end{Verbatim}


    \begin{Verbatim}[commandchars=\\\{\}]
('Honda', 'City', 'EX 1.5', 2018, 69650.0)

    \end{Verbatim}

    Uma \emph{tupla} nula ou vazia é representada apenas por um par de
parênteses.

    \begin{Verbatim}[commandchars=\\\{\}]
{\color{incolor}In [{\color{incolor}9}]:} \PY{n}{nada} \PY{o}{=} \PY{p}{(}\PY{p}{)}
        \PY{n+nb}{print}\PY{p}{(}\PY{n+nb}{type}\PY{p}{(}\PY{n}{nada}\PY{p}{)}\PY{p}{,} \PY{n}{nada}\PY{p}{)}
\end{Verbatim}


    \begin{Verbatim}[commandchars=\\\{\}]
<class 'tuple'> ()

    \end{Verbatim}

    A criação de uma \emph{tupla} com um único elemento já requer um cuidado
extra. Veja o que acontece...

    \begin{Verbatim}[commandchars=\\\{\}]
{\color{incolor}In [{\color{incolor}11}]:} \PY{n}{marca} \PY{o}{=} \PY{p}{(}\PY{l+s+s1}{\PYZsq{}}\PY{l+s+s1}{Honda}\PY{l+s+s1}{\PYZsq{}}\PY{p}{)}  \PY{c+c1}{\PYZsh{} isto é considerado uso normal de parênteses}
         \PY{n+nb}{print}\PY{p}{(}\PY{n+nb}{type}\PY{p}{(}\PY{n}{marca}\PY{p}{)}\PY{p}{)}
\end{Verbatim}


    \begin{Verbatim}[commandchars=\\\{\}]
<class 'str'>

    \end{Verbatim}

    \begin{Verbatim}[commandchars=\\\{\}]
{\color{incolor}In [{\color{incolor}12}]:} \PY{n}{marca} \PY{o}{=} \PY{p}{(}\PY{l+s+s1}{\PYZsq{}}\PY{l+s+s1}{Honda}\PY{l+s+s1}{\PYZsq{}}\PY{p}{,}\PY{p}{)}  \PY{c+c1}{\PYZsh{} uma vírgula antes do parêntese direito faz com que a expressão seja reconhecida corretamente}
         \PY{n+nb}{print}\PY{p}{(}\PY{n+nb}{type}\PY{p}{(}\PY{n}{marca}\PY{p}{)}\PY{p}{)}
\end{Verbatim}


    \begin{Verbatim}[commandchars=\\\{\}]
<class 'tuple'>

    \end{Verbatim}

    \subsection{Como ``desempacotar'' uma
tupla}\label{como-desempacotar-uma-tupla}

Numa atribuição é possível associar os elementos de uma tupla colocada
do lado direito a uma tupla de variáveis colocada do lado esquerdo.

    \begin{Verbatim}[commandchars=\\\{\}]
{\color{incolor}In [{\color{incolor}14}]:} \PY{n}{marca}\PY{p}{,} \PY{n}{modelo}\PY{p}{,} \PY{n}{motor}\PY{p}{,} \PY{n}{ano}\PY{p}{,} \PY{n}{valor} \PY{o}{=} \PY{n}{auto}
         \PY{n+nb}{print}\PY{p}{(}\PY{n}{ano}\PY{p}{,} \PY{n}{marca}\PY{p}{,} \PY{n}{modelo}\PY{p}{,} \PY{n}{motor}\PY{p}{,} \PY{n}{valor}\PY{p}{)}
\end{Verbatim}


    \begin{Verbatim}[commandchars=\\\{\}]
2016 Honda City DX 1.5 52500.0

    \end{Verbatim}

    \begin{Verbatim}[commandchars=\\\{\}]
{\color{incolor}In [{\color{incolor}15}]:} \PY{n}{marca}\PY{p}{,} \PY{n}{modelo}\PY{p}{,} \PY{o}{*}\PY{n}{outros} \PY{o}{=} \PY{n}{auto}
         \PY{n+nb}{print}\PY{p}{(}\PY{n}{marca}\PY{p}{,} \PY{n}{modelo}\PY{p}{,} \PY{n}{outros}\PY{p}{)}
\end{Verbatim}


    \begin{Verbatim}[commandchars=\\\{\}]
Honda City ['DX 1.5', 2016, 52500.0]

    \end{Verbatim}

    Também é possível usar uma tupla de expressões do lado direito do
operador de atribuição. Nesse caso, as expressões são avaliadas da
esquerda para a direita e seus resultados atribuídos às variáveis da
tupla do lado esquerdo, também da esquerda para a direita.

Essa propriedade permite uma forma simples para permutar os valores de
duas variáveis ou para ``rodar'' os valores de três ou mais variáveis.

    \begin{Verbatim}[commandchars=\\\{\}]
{\color{incolor}In [{\color{incolor} }]:} \PY{n}{a} \PY{o}{=} \PY{l+m+mi}{10}
        \PY{n}{b} \PY{o}{=} \PY{l+m+mi}{20}
        \PY{n}{a}\PY{p}{,} \PY{n}{b} \PY{o}{=} \PY{n}{b}\PY{p}{,} \PY{n}{a}
        \PY{n+nb}{print}\PY{p}{(}\PY{l+s+s1}{\PYZsq{}}\PY{l+s+s1}{a =}\PY{l+s+s1}{\PYZsq{}}\PY{p}{,} \PY{n}{a}\PY{p}{,} \PY{l+s+s1}{\PYZsq{}}\PY{l+s+s1}{  b =}\PY{l+s+s1}{\PYZsq{}}\PY{p}{,} \PY{n}{b}\PY{p}{)}
\end{Verbatim}


    \begin{Verbatim}[commandchars=\\\{\}]
{\color{incolor}In [{\color{incolor} }]:} \PY{n}{a}\PY{p}{,} \PY{n}{b}\PY{p}{,} \PY{n}{a} \PY{o}{=} \PY{l+m+mi}{100}\PY{p}{,} \PY{l+m+mi}{200}\PY{p}{,} \PY{l+m+mi}{300}  \PY{c+c1}{\PYZsh{} a aparece duas vezes na lista de variáveis}
        \PY{n+nb}{print}\PY{p}{(}\PY{l+s+s1}{\PYZsq{}}\PY{l+s+s1}{a =}\PY{l+s+s1}{\PYZsq{}}\PY{p}{,} \PY{n}{a}\PY{p}{,} \PY{l+s+s1}{\PYZsq{}}\PY{l+s+s1}{  b =}\PY{l+s+s1}{\PYZsq{}}\PY{p}{,} \PY{n}{b}\PY{p}{)}
\end{Verbatim}


    \begin{Verbatim}[commandchars=\\\{\}]
{\color{incolor}In [{\color{incolor} }]:} \PY{n}{a}\PY{p}{,} \PY{n}{b}\PY{p}{,} \PY{n}{c} \PY{o}{=} \PY{l+m+mi}{100}\PY{p}{,} \PY{l+m+mi}{200}\PY{p}{,} \PY{l+m+mi}{300}
        \PY{n+nb}{print}\PY{p}{(}\PY{l+s+s1}{\PYZsq{}}\PY{l+s+s1}{a =}\PY{l+s+s1}{\PYZsq{}}\PY{p}{,} \PY{n}{a}\PY{p}{,} \PY{l+s+s1}{\PYZsq{}}\PY{l+s+s1}{  b =}\PY{l+s+s1}{\PYZsq{}}\PY{p}{,} \PY{n}{b}\PY{p}{,} \PY{l+s+s1}{\PYZsq{}}\PY{l+s+s1}{  c =}\PY{l+s+s1}{\PYZsq{}}\PY{p}{,} \PY{n}{c}\PY{p}{)}
        
        \PY{n}{a}\PY{p}{,} \PY{n}{b}\PY{p}{,} \PY{n}{c} \PY{o}{=} \PY{n}{b}\PY{p}{,} \PY{n}{c}\PY{p}{,} \PY{n}{a}
        \PY{n+nb}{print}\PY{p}{(}\PY{l+s+s1}{\PYZsq{}}\PY{l+s+s1}{a =}\PY{l+s+s1}{\PYZsq{}}\PY{p}{,} \PY{n}{a}\PY{p}{,} \PY{l+s+s1}{\PYZsq{}}\PY{l+s+s1}{  b =}\PY{l+s+s1}{\PYZsq{}}\PY{p}{,} \PY{n}{b}\PY{p}{,} \PY{l+s+s1}{\PYZsq{}}\PY{l+s+s1}{  c =}\PY{l+s+s1}{\PYZsq{}}\PY{p}{,} \PY{n}{c}\PY{p}{)}
\end{Verbatim}


    O módulo \texttt{operator} oferece o extrator \texttt{itemgetter} que
pode ser usado para obter um ou mais elementos de uma tupla. Na sua
forma mais simples, aplicando-se \texttt{itemgetter} a um valor que
representa um índice, ele gera uma função capaz de extrair o elemento
naquela posição em uma sequência. Por exemplo, no caso de uma tupla...

    \begin{Verbatim}[commandchars=\\\{\}]
{\color{incolor}In [{\color{incolor}16}]:} \PY{k+kn}{from} \PY{n+nn}{operator} \PY{k}{import} \PY{n}{itemgetter}
         
         \PY{n}{t} \PY{o}{=} \PY{p}{(}\PY{l+m+mi}{10}\PY{p}{,} \PY{l+m+mi}{20}\PY{p}{,} \PY{l+m+mi}{30}\PY{p}{)}
         \PY{n}{itemgetter}\PY{p}{(}\PY{l+m+mi}{2}\PY{p}{)}\PY{p}{(}\PY{n}{t}\PY{p}{)}
\end{Verbatim}


\begin{Verbatim}[commandchars=\\\{\}]
{\color{outcolor}Out[{\color{outcolor}16}]:} 30
\end{Verbatim}
            
    \subsection{Como criar tuplas a partir de listas
``paralelas''}\label{como-criar-tuplas-a-partir-de-listas-paralelas}

A função \emph{zip} retorna um iterador de tuplas, onde a i-ésima tupla
contém o i-ésimo elemento de cada uma das sequências ou iteradores
passados como argumentos.\\
O iterador para quando o argumento ``mais curto'' se esgota.\\
Se for passado um único argumento, ela retorna um iterador de
1-tuplas.\\
Sem argumentos, ela retorna um iterador vazio.\\
Os argumentos iteráveis são avaliados da esquerda para a direita.

    \begin{Verbatim}[commandchars=\\\{\}]
{\color{incolor}In [{\color{incolor}3}]:} \PY{n}{la} \PY{o}{=} \PY{p}{[}\PY{l+m+mi}{1}\PY{p}{,} \PY{l+m+mi}{2}\PY{p}{,} \PY{l+m+mi}{3}\PY{p}{]}
        \PY{n}{lb} \PY{o}{=} \PY{p}{[}\PY{l+m+mi}{4}\PY{p}{,} \PY{l+m+mi}{5}\PY{p}{,} \PY{l+m+mi}{6}\PY{p}{]}
        \PY{n}{lc} \PY{o}{=} \PY{p}{[}\PY{l+m+mi}{7}\PY{p}{,} \PY{l+m+mi}{8}\PY{p}{,} \PY{l+m+mi}{9}\PY{p}{]}
        \PY{n}{lz} \PY{o}{=} \PY{n+nb}{zip}\PY{p}{(}\PY{n}{la}\PY{p}{,} \PY{n}{lb}\PY{p}{,} \PY{n}{lc}\PY{p}{)}
        \PY{n+nb}{list}\PY{p}{(}\PY{n}{lz}\PY{p}{)}
\end{Verbatim}


\begin{Verbatim}[commandchars=\\\{\}]
{\color{outcolor}Out[{\color{outcolor}3}]:} [(1, 4, 7), (2, 5, 8), (3, 6, 9)]
\end{Verbatim}
            
    Quando combinados, a função \emph{zip}, o comando \emph{for} e a
decomposição de tuplas permitem manipular várias sequências em paralelo,
tratando os elementos correspondentes sem o uso explícito de índices.\\
Por exemplo, ...

    \begin{Verbatim}[commandchars=\\\{\}]
{\color{incolor}In [{\color{incolor}5}]:} \PY{n}{ls} \PY{o}{=} \PY{p}{[}\PY{p}{]}
        \PY{k}{for} \PY{n}{a}\PY{p}{,} \PY{n}{b}\PY{p}{,} \PY{n}{c} \PY{o+ow}{in} \PY{n+nb}{zip}\PY{p}{(}\PY{n}{la}\PY{p}{,} \PY{n}{lb}\PY{p}{,} \PY{n}{lc}\PY{p}{)}\PY{p}{:}
            \PY{n}{ls} \PY{o}{+}\PY{o}{=} \PY{p}{[}\PY{n}{a} \PY{o}{+} \PY{n}{b} \PY{o}{+} \PY{n}{c}\PY{p}{]}
        \PY{n}{ls}
\end{Verbatim}


\begin{Verbatim}[commandchars=\\\{\}]
{\color{outcolor}Out[{\color{outcolor}5}]:} [12, 15, 18]
\end{Verbatim}
            
    \subsubsection{Exemplo: Cálculo do produto interno de duas
listas}\label{exemplo-cuxe1lculo-do-produto-interno-de-duas-listas}

Dadas duas listas de elementos numéricos, calcular seu produto interno.

Este problema foi resolvido como parte do exemplo \emph{Cálculo da média
ponderada} na \emph{Aula 11}. Aqui nos interessam apenas as linhas 9-11.

O uso de \emph{zip} para criar um iterador combinando as listas
\emph{notas} e \emph{pesos} e depois a extração das duplas uma a uma,
permitem dispensar a manipulação de índices que era necessária na
implementação anterior.

    \begin{Verbatim}[commandchars=\\\{\}]
{\color{incolor}In [{\color{incolor}17}]:} \PY{k+kn}{from} \PY{n+nn}{random} \PY{k}{import} \PY{n}{choice}
         
         \PY{n}{pesos} \PY{o}{=} \PY{p}{[}\PY{n}{choice}\PY{p}{(}\PY{n+nb}{range}\PY{p}{(}\PY{l+m+mi}{1}\PY{p}{,} \PY{l+m+mi}{4}\PY{p}{)}\PY{p}{)} \PY{k}{for} \PY{n}{\PYZus{}} \PY{o+ow}{in} \PY{n+nb}{range}\PY{p}{(}\PY{l+m+mi}{10}\PY{p}{)}\PY{p}{]}
         \PY{n+nb}{print}\PY{p}{(}\PY{l+s+s1}{\PYZsq{}}\PY{l+s+s1}{pesos}\PY{l+s+s1}{\PYZsq{}}\PY{p}{,} \PY{n}{pesos}\PY{p}{)}
         
         \PY{n}{notas} \PY{o}{=} \PY{p}{[}\PY{n}{choice}\PY{p}{(}\PY{n+nb}{range}\PY{p}{(}\PY{l+m+mi}{11}\PY{p}{)}\PY{p}{)} \PY{k}{for} \PY{n}{\PYZus{}} \PY{o+ow}{in} \PY{n+nb}{range}\PY{p}{(}\PY{l+m+mi}{10}\PY{p}{)}\PY{p}{]}
         \PY{n+nb}{print}\PY{p}{(}\PY{l+s+s1}{\PYZsq{}}\PY{l+s+s1}{notas}\PY{l+s+s1}{\PYZsq{}}\PY{p}{,} \PY{n}{notas}\PY{p}{)}
         
         \PY{n}{total} \PY{o}{=} \PY{l+m+mi}{0}
         \PY{k}{for} \PY{n}{nota}\PY{p}{,} \PY{n}{peso} \PY{o+ow}{in} \PY{n+nb}{zip}\PY{p}{(}\PY{n}{notas}\PY{p}{,} \PY{n}{pesos}\PY{p}{)}\PY{p}{:}
             \PY{n}{total} \PY{o}{+}\PY{o}{=} \PY{n}{nota} \PY{o}{*} \PY{n}{peso}
         
         \PY{n}{média} \PY{o}{=} \PY{n}{total} \PY{o}{/} \PY{n+nb}{sum}\PY{p}{(}\PY{n}{pesos}\PY{p}{)}
         \PY{n+nb}{print}\PY{p}{(}\PY{n}{f}\PY{l+s+s1}{\PYZsq{}}\PY{l+s+s1}{total }\PY{l+s+si}{\PYZob{}total\PYZcb{}}\PY{l+s+s1}{   média }\PY{l+s+si}{\PYZob{}média:.1f\PYZcb{}}\PY{l+s+s1}{\PYZsq{}}\PY{p}{)}
\end{Verbatim}


    \begin{Verbatim}[commandchars=\\\{\}]
pesos [3, 1, 2, 2, 3, 2, 3, 1, 1, 1]
notas [3, 3, 7, 5, 5, 5, 0, 8, 4, 7]
total 80   média 4.2

    \end{Verbatim}

    \subsection{\texorpdfstring{\emph{Muito cuidado com
aliasing}}{Muito cuidado com aliasing}}\label{muito-cuidado-com-aliasing}

\textbf{Python tuples: immutable but potentially changing}\\
Think labels, not boxes\\
by Luciano Ramalho

    Tweeddledee e Tweeddledum são os gêmeos que Alice encontra em "Através
do Espelho". Vamos chamá-los Dee e Dum e representá-los como
\emph{tuplas} com sua data de nascimento e habilidades.

    \begin{Verbatim}[commandchars=\\\{\}]
{\color{incolor}In [{\color{incolor}18}]:} \PY{n}{dee} \PY{o}{=} \PY{p}{(}\PY{l+s+s1}{\PYZsq{}}\PY{l+s+s1}{1861\PYZhy{}10\PYZhy{}23}\PY{l+s+s1}{\PYZsq{}}\PY{p}{,} \PY{p}{[}\PY{l+s+s1}{\PYZsq{}}\PY{l+s+s1}{poesia}\PY{l+s+s1}{\PYZsq{}}\PY{p}{,} \PY{l+s+s1}{\PYZsq{}}\PY{l+s+s1}{briga\PYZhy{}simulada}\PY{l+s+s1}{\PYZsq{}}\PY{p}{]}\PY{p}{)}
         \PY{n}{dum} \PY{o}{=} \PY{p}{(}\PY{l+s+s1}{\PYZsq{}}\PY{l+s+s1}{1861\PYZhy{}10\PYZhy{}23}\PY{l+s+s1}{\PYZsq{}}\PY{p}{,} \PY{p}{[}\PY{l+s+s1}{\PYZsq{}}\PY{l+s+s1}{poesia}\PY{l+s+s1}{\PYZsq{}}\PY{p}{,} \PY{l+s+s1}{\PYZsq{}}\PY{l+s+s1}{briga\PYZhy{}simulada}\PY{l+s+s1}{\PYZsq{}}\PY{p}{]}\PY{p}{)}
\end{Verbatim}


    Como Dee e Dum são gêmeos, suas representações são iguais, embora não
sejam uma só.

    \begin{Verbatim}[commandchars=\\\{\}]
{\color{incolor}In [{\color{incolor}19}]:} \PY{n}{dum} \PY{o}{==} \PY{n}{dee}
\end{Verbatim}


\begin{Verbatim}[commandchars=\\\{\}]
{\color{outcolor}Out[{\color{outcolor}19}]:} True
\end{Verbatim}
            
    \begin{Verbatim}[commandchars=\\\{\}]
{\color{incolor}In [{\color{incolor}20}]:} \PY{n}{dum} \PY{o+ow}{is} \PY{n}{dee}
\end{Verbatim}


\begin{Verbatim}[commandchars=\\\{\}]
{\color{outcolor}Out[{\color{outcolor}20}]:} False
\end{Verbatim}
            
    \begin{Verbatim}[commandchars=\\\{\}]
{\color{incolor}In [{\color{incolor}21}]:} \PY{n+nb}{id}\PY{p}{(}\PY{n}{dum}\PY{p}{)}\PY{p}{,} \PY{n+nb}{id}\PY{p}{(}\PY{n}{dee}\PY{p}{)}
\end{Verbatim}


\begin{Verbatim}[commandchars=\\\{\}]
{\color{outcolor}Out[{\color{outcolor}21}]:} (4521652296, 4521417544)
\end{Verbatim}
            
    Por outro lado, como ambos estão representados como tuplas, que são
objetos imutáveis, não deve ser possível destruir a igualdade entre
eles, certo?

    Vamos criar um \emph{clone} de Dum.

    \begin{Verbatim}[commandchars=\\\{\}]
{\color{incolor}In [{\color{incolor}22}]:} \PY{n}{doom} \PY{o}{=} \PY{n}{dum}
\end{Verbatim}


    \begin{Verbatim}[commandchars=\\\{\}]
{\color{incolor}In [{\color{incolor}23}]:} \PY{n}{doom} \PY{o}{==} \PY{n}{dum}
\end{Verbatim}


\begin{Verbatim}[commandchars=\\\{\}]
{\color{outcolor}Out[{\color{outcolor}23}]:} True
\end{Verbatim}
            
    \begin{Verbatim}[commandchars=\\\{\}]
{\color{incolor}In [{\color{incolor}24}]:} \PY{n}{doom} \PY{o+ow}{is} \PY{n}{dum}
\end{Verbatim}


\begin{Verbatim}[commandchars=\\\{\}]
{\color{outcolor}Out[{\color{outcolor}24}]:} True
\end{Verbatim}
            
    \begin{Verbatim}[commandchars=\\\{\}]
{\color{incolor}In [{\color{incolor}25}]:} \PY{n+nb}{id}\PY{p}{(}\PY{n}{doom}\PY{p}{)}\PY{p}{,} \PY{n+nb}{id}\PY{p}{(}\PY{n}{dum}\PY{p}{)}
\end{Verbatim}


\begin{Verbatim}[commandchars=\\\{\}]
{\color{outcolor}Out[{\color{outcolor}25}]:} (4521652296, 4521652296)
\end{Verbatim}
            
    Suponha agora que Doom tenha se tornado \emph{rapper}. Vamos acrescentar
isso às suas habilidades. Estas estão representadas em
\texttt{doom{[}1{]}}, certo?

    \begin{Verbatim}[commandchars=\\\{\}]
{\color{incolor}In [{\color{incolor}26}]:} \PY{n}{doom}\PY{p}{[}\PY{l+m+mi}{1}\PY{p}{]}
\end{Verbatim}


\begin{Verbatim}[commandchars=\\\{\}]
{\color{outcolor}Out[{\color{outcolor}26}]:} ['poesia', 'briga-simulada']
\end{Verbatim}
            
    \begin{Verbatim}[commandchars=\\\{\}]
{\color{incolor}In [{\color{incolor}27}]:} \PY{n}{doom}\PY{p}{[}\PY{l+m+mi}{1}\PY{p}{]}\PY{o}{.}\PY{n}{append}\PY{p}{(}\PY{l+s+s1}{\PYZsq{}}\PY{l+s+s1}{rap}\PY{l+s+s1}{\PYZsq{}}\PY{p}{)}
         \PY{n}{doom}
\end{Verbatim}


\begin{Verbatim}[commandchars=\\\{\}]
{\color{outcolor}Out[{\color{outcolor}27}]:} ('1861-10-23', ['poesia', 'briga-simulada', 'rap'])
\end{Verbatim}
            
    Vamos exibir as \emph{tuplas} que representam Dee, Dum e Doom novamente.

    \begin{Verbatim}[commandchars=\\\{\}]
{\color{incolor}In [{\color{incolor}28}]:} \PY{n}{dee}
         \PY{n}{dum}
         \PY{n}{doom}
\end{Verbatim}


\begin{Verbatim}[commandchars=\\\{\}]
{\color{outcolor}Out[{\color{outcolor}28}]:} ('1861-10-23', ['poesia', 'briga-simulada'])
\end{Verbatim}
            
\begin{Verbatim}[commandchars=\\\{\}]
{\color{outcolor}Out[{\color{outcolor}28}]:} ('1861-10-23', ['poesia', 'briga-simulada', 'rap'])
\end{Verbatim}
            
\begin{Verbatim}[commandchars=\\\{\}]
{\color{outcolor}Out[{\color{outcolor}28}]:} ('1861-10-23', ['poesia', 'briga-simulada', 'rap'])
\end{Verbatim}
            
    \begin{Verbatim}[commandchars=\\\{\}]
{\color{incolor}In [{\color{incolor}29}]:} \PY{n}{dum} \PY{o}{==} \PY{n}{dee}
\end{Verbatim}


\begin{Verbatim}[commandchars=\\\{\}]
{\color{outcolor}Out[{\color{outcolor}29}]:} False
\end{Verbatim}
            
    Ooops! \texttt{dum} e \texttt{dee} deixaram de ser iguais!\\
Mas tuplas não são imutáveis?

\begin{itemize}
\tightlist
\item
  Sim, isto é, \textbf{nem sempre}...
\end{itemize}

    \paragraph{Moral da história}\label{moral-da-histuxf3ria}

\begin{quote}
Nunca use um objeto mutável como componente de um objeto
``teoricamente'' imutável.\\
Você pode se surpreender!
\end{quote}

    \subsection{Named tuples}\label{named-tuples}

\emph{Named tuples} representam uma extensão ao modelo de \emph{tuplas}
buscando atacar alguns dos problemas acima.\\
Elas são sequências ordenadas, ``imutáveis'' e iteráveis como
\emph{tuplas} normais, mas seus elementos podem também ser referenciados
por nome, como os campos dos \emph{records} de outras linguagens de alto
nível.

    \begin{Verbatim}[commandchars=\\\{\}]
{\color{incolor}In [{\color{incolor}30}]:} \PY{k+kn}{from} \PY{n+nn}{collections} \PY{k}{import} \PY{n}{namedtuple}
         
         \PY{n}{Auto} \PY{o}{=} \PY{n}{namedtuple}\PY{p}{(}\PY{l+s+s1}{\PYZsq{}}\PY{l+s+s1}{Auto}\PY{l+s+s1}{\PYZsq{}}\PY{p}{,} \PY{l+s+s1}{\PYZsq{}}\PY{l+s+s1}{marca modelo motor ano valor}\PY{l+s+s1}{\PYZsq{}}\PY{p}{)}
         
         \PY{n}{auto} \PY{o}{=} \PY{n}{Auto}\PY{p}{(}\PY{l+s+s1}{\PYZsq{}}\PY{l+s+s1}{Honda}\PY{l+s+s1}{\PYZsq{}}\PY{p}{,} \PY{l+s+s1}{\PYZsq{}}\PY{l+s+s1}{City}\PY{l+s+s1}{\PYZsq{}}\PY{p}{,} \PY{l+s+s1}{\PYZsq{}}\PY{l+s+s1}{DX 1.5}\PY{l+s+s1}{\PYZsq{}}\PY{p}{,} \PY{l+m+mi}{2016}\PY{p}{,} \PY{l+m+mf}{52500.0}\PY{p}{)}
         \PY{n+nb}{print}\PY{p}{(}\PY{n}{auto}\PY{p}{)}
\end{Verbatim}


    \begin{Verbatim}[commandchars=\\\{\}]
Auto(marca='Honda', modelo='City', motor='DX 1.5', ano=2016, valor=52500.0)

    \end{Verbatim}

    \begin{Verbatim}[commandchars=\\\{\}]
{\color{incolor}In [{\color{incolor}32}]:} \PY{n}{auto}\PY{o}{.}\PY{n}{ano} \PY{o}{=} \PY{l+m+mi}{2018}
\end{Verbatim}


    \begin{Verbatim}[commandchars=\\\{\}]

        ---------------------------------------------------------------------------

        AttributeError                            Traceback (most recent call last)

        <ipython-input-32-d0999d039f0b> in <module>()
    ----> 1 auto.ano = 2018
    

        AttributeError: can't set attribute

    \end{Verbatim}

    \begin{Verbatim}[commandchars=\\\{\}]
{\color{incolor}In [{\color{incolor}33}]:} \PY{n+nb}{print}\PY{p}{(}\PY{n}{auto}\PY{o}{.}\PY{n}{valor}\PY{p}{,} \PY{n}{auto}\PY{p}{[}\PY{l+m+mi}{4}\PY{p}{]}\PY{p}{)}
\end{Verbatim}


    \begin{Verbatim}[commandchars=\\\{\}]
52500.0 52500.0

    \end{Verbatim}

    \subsubsection{No entanto, o problema com aliasing
permanece}\label{no-entanto-o-problema-com-aliasing-permanece}

    Vamos criar uma \emph{namedtuple} para conter os dados pessoais dos
gêmeos e depois criar as respectivas entradas.

    \begin{Verbatim}[commandchars=\\\{\}]
{\color{incolor}In [{\color{incolor}44}]:} \PY{n}{Bio} \PY{o}{=} \PY{n}{namedtuple}\PY{p}{(}\PY{l+s+s1}{\PYZsq{}}\PY{l+s+s1}{Bio}\PY{l+s+s1}{\PYZsq{}}\PY{p}{,} \PY{l+s+s1}{\PYZsq{}}\PY{l+s+s1}{aniversário habilidades}\PY{l+s+s1}{\PYZsq{}}\PY{p}{)}
\end{Verbatim}


    \begin{Verbatim}[commandchars=\\\{\}]
{\color{incolor}In [{\color{incolor}45}]:} \PY{n}{dee} \PY{o}{=} \PY{n}{Bio}\PY{p}{(}\PY{l+s+s1}{\PYZsq{}}\PY{l+s+s1}{1861\PYZhy{}10\PYZhy{}23}\PY{l+s+s1}{\PYZsq{}}\PY{p}{,} \PY{p}{[}\PY{l+s+s1}{\PYZsq{}}\PY{l+s+s1}{poesia}\PY{l+s+s1}{\PYZsq{}}\PY{p}{,} \PY{l+s+s1}{\PYZsq{}}\PY{l+s+s1}{briga\PYZhy{}simulada}\PY{l+s+s1}{\PYZsq{}}\PY{p}{]}\PY{p}{)}
         \PY{n}{dee}
\end{Verbatim}


\begin{Verbatim}[commandchars=\\\{\}]
{\color{outcolor}Out[{\color{outcolor}45}]:} Bio(aniversário='1861-10-23', habilidades=['poesia', 'briga-simulada'])
\end{Verbatim}
            
    \begin{Verbatim}[commandchars=\\\{\}]
{\color{incolor}In [{\color{incolor}46}]:} \PY{n}{dum} \PY{o}{=} \PY{n}{Bio}\PY{p}{(}\PY{l+s+s1}{\PYZsq{}}\PY{l+s+s1}{1861\PYZhy{}10\PYZhy{}23}\PY{l+s+s1}{\PYZsq{}}\PY{p}{,} \PY{p}{[}\PY{l+s+s1}{\PYZsq{}}\PY{l+s+s1}{poesia}\PY{l+s+s1}{\PYZsq{}}\PY{p}{,} \PY{l+s+s1}{\PYZsq{}}\PY{l+s+s1}{briga\PYZhy{}simulada}\PY{l+s+s1}{\PYZsq{}}\PY{p}{]}\PY{p}{)}
         \PY{n}{dum}
\end{Verbatim}


\begin{Verbatim}[commandchars=\\\{\}]
{\color{outcolor}Out[{\color{outcolor}46}]:} Bio(aniversário='1861-10-23', habilidades=['poesia', 'briga-simulada'])
\end{Verbatim}
            
    Vamos verificar a igualdade e a distinção das \emph{namedtuples}.

    \begin{Verbatim}[commandchars=\\\{\}]
{\color{incolor}In [{\color{incolor}47}]:} \PY{n}{dee} \PY{o}{==} \PY{n}{dum}
\end{Verbatim}


\begin{Verbatim}[commandchars=\\\{\}]
{\color{outcolor}Out[{\color{outcolor}47}]:} True
\end{Verbatim}
            
    \begin{Verbatim}[commandchars=\\\{\}]
{\color{incolor}In [{\color{incolor}48}]:} \PY{n}{dee} \PY{o+ow}{is} \PY{n}{dum}
\end{Verbatim}


\begin{Verbatim}[commandchars=\\\{\}]
{\color{outcolor}Out[{\color{outcolor}48}]:} False
\end{Verbatim}
            
    \begin{Verbatim}[commandchars=\\\{\}]
{\color{incolor}In [{\color{incolor}49}]:} \PY{n+nb}{id}\PY{p}{(}\PY{n}{dum}\PY{p}{)}\PY{p}{,} \PY{n+nb}{id}\PY{p}{(}\PY{n}{dee}\PY{p}{)}
\end{Verbatim}


\begin{Verbatim}[commandchars=\\\{\}]
{\color{outcolor}Out[{\color{outcolor}49}]:} (4523418032, 4523417744)
\end{Verbatim}
            
    Vamos criar Doom, o clone de Dum.

    \begin{Verbatim}[commandchars=\\\{\}]
{\color{incolor}In [{\color{incolor}50}]:} \PY{n}{doom} \PY{o}{=} \PY{n}{dum}
         \PY{n}{doom}
\end{Verbatim}


\begin{Verbatim}[commandchars=\\\{\}]
{\color{outcolor}Out[{\color{outcolor}50}]:} Bio(aniversário='1861-10-23', habilidades=['poesia', 'briga-simulada'])
\end{Verbatim}
            
    \begin{Verbatim}[commandchars=\\\{\}]
{\color{incolor}In [{\color{incolor}51}]:} \PY{n}{doom} \PY{o}{==} \PY{n}{dum}
\end{Verbatim}


\begin{Verbatim}[commandchars=\\\{\}]
{\color{outcolor}Out[{\color{outcolor}51}]:} True
\end{Verbatim}
            
    \begin{Verbatim}[commandchars=\\\{\}]
{\color{incolor}In [{\color{incolor}52}]:} \PY{n}{doom} \PY{o+ow}{is} \PY{n}{dum}
\end{Verbatim}


\begin{Verbatim}[commandchars=\\\{\}]
{\color{outcolor}Out[{\color{outcolor}52}]:} True
\end{Verbatim}
            
    Vamos acrescentar \texttt{rap} às habilidades de Doom.

    \begin{Verbatim}[commandchars=\\\{\}]
{\color{incolor}In [{\color{incolor}53}]:} \PY{n}{doom}\PY{p}{[}\PY{l+m+mi}{1}\PY{p}{]}\PY{o}{.}\PY{n}{append}\PY{p}{(}\PY{l+s+s1}{\PYZsq{}}\PY{l+s+s1}{rap}\PY{l+s+s1}{\PYZsq{}}\PY{p}{)}
         \PY{n}{doom}
\end{Verbatim}


\begin{Verbatim}[commandchars=\\\{\}]
{\color{outcolor}Out[{\color{outcolor}53}]:} Bio(aniversário='1861-10-23', habilidades=['poesia', 'briga-simulada', 'rap'])
\end{Verbatim}
            
    Como \texttt{doom} e \texttt{dum} são ``rótulos'' que estão ``colados''
no mesmo objeto, isto também afeta as habilidades de Dum que, por
estarem representadas numa \emph{namedtuple}, \textbf{deveriam ser
imutáveis}.

    \begin{Verbatim}[commandchars=\\\{\}]
{\color{incolor}In [{\color{incolor}54}]:} \PY{n}{dum}
\end{Verbatim}


\begin{Verbatim}[commandchars=\\\{\}]
{\color{outcolor}Out[{\color{outcolor}54}]:} Bio(aniversário='1861-10-23', habilidades=['poesia', 'briga-simulada', 'rap'])
\end{Verbatim}
            
    Como \texttt{dee} continua o mesmo, \texttt{dum} e \texttt{dee} deixaram
de ser iguais, o que, teoricamente, não poderia acontecer.

    \begin{Verbatim}[commandchars=\\\{\}]
{\color{incolor}In [{\color{incolor}55}]:} \PY{n}{dee}
\end{Verbatim}


\begin{Verbatim}[commandchars=\\\{\}]
{\color{outcolor}Out[{\color{outcolor}55}]:} Bio(aniversário='1861-10-23', habilidades=['poesia', 'briga-simulada'])
\end{Verbatim}
            
    \begin{Verbatim}[commandchars=\\\{\}]
{\color{incolor}In [{\color{incolor}56}]:} \PY{n}{dee} \PY{o}{==} \PY{n}{dum}
\end{Verbatim}


\begin{Verbatim}[commandchars=\\\{\}]
{\color{outcolor}Out[{\color{outcolor}56}]:} False
\end{Verbatim}
            

    % Add a bibliography block to the postdoc
    
    
    
    \end{document}
